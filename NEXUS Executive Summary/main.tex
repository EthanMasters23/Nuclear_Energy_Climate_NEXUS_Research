 \documentclass[11,5 pt]{article}
\usepackage{natbib}
\usepackage{tikz,lipsum,lmodern}
\usepackage[most]{tcolorbox}
\usepackage{varwidth}
\usepackage{geometry}
\usepackage{url}
\usepackage[nottoc]{tocbibind}
\usepackage{fancyhdr}
\usepackage{enumerate}
\usepackage[utf8]{inputenc}
\usepackage{float}
\usepackage{caption}
\usepackage{subcaption}
\usepackage{cancel}
\usepackage{graphicx}               % Necessary to use \scalebox
\usepackage{textcomp}
\usepackage{amsmath}
\usepackage{dsfont}
\usepackage{cool}
\usepackage{enumitem}
\usepackage{csquotes}
\usepackage{comment}
\usepackage[sorting=none]{biblatex}
\bibliography{references}
\usepackage{tabularx}
\usepackage{chngcntr}
\usepackage[framed,numbered,autolinebreaks,useliterate]{mcode}
\usepackage{multicol}
\usepackage{enumitem,kantlipsum}
\usepackage[hidelinks]{hyperref}


\geometry{
 a4paper,
 total={160mm,223mm},
 left=25mm,
 top=42mm,
}

%% Definiciones de cuadros

\definecolor{anibalMorao}{HTML}{3366CC}

\newtcolorbox[auto counter,number within=section]{importanteBoxNumerado}[2][]{enhanced, colback=blue!5,colframe=blue!50,boxrule=0.4mm, attach boxed title to top left={xshift=1cm,yshift*=1mm-\tcboxedtitleheight}, varwidth boxed title*=-3cm, boxed title style={frame code={ \path[fill=tcbcolback!30!black] ([yshift=-1mm,xshift=-1mm]frame.north west) arc[start angle=0,end angle=180,radius=1mm] ([yshift=-1mm,xshift=1mm]frame.north east) arc[start angle=180,end angle=0,radius=1mm]; \path[left color=tcbcolback!60!black,right color=tcbcolback!60!black, middle color=tcbcolback!80!black] ([xshift=-2mm]frame.north west) -- ([xshift=2mm]frame.north east) [rounded corners=1mm]-- ([xshift=1mm,yshift=-1mm]frame.north east)
-- (frame.south east) -- (frame.south west)
-- ([xshift=-1mm,yshift=-1mm]frame.north west) [sharp corners]-- cycle; },interior engine=empty, }, fonttitle=\bfseries, title=Importante~\thetcbcounter. #2,#1}

\newtcolorbox[auto counter,number within=section]{destacadoBox}[2][]{enhanced,
colback=blue!5,colframe=blue!50,boxrule=0.4mm, attach boxed title to top left={xshift=1cm,yshift*=1mm-\tcboxedtitleheight}, varwidth boxed title*=-3cm, boxed title style={frame code={ \path[fill=tcbcolback!30!black] ([yshift=-1mm,xshift=-1mm]frame.north west) arc[start angle=0,end angle=180,radius=1mm] ([yshift=-1mm,xshift=1mm]frame.north east) arc[start angle=180,end angle=0,radius=1mm]; \path[left color=tcbcolback!60!black,right color=tcbcolback!60!black, middle color=tcbcolback!80!black] ([xshift=-2mm]frame.north west) -- ([xshift=2mm]frame.north east) [rounded corners=1mm]-- ([xshift=1mm,yshift=-1mm]frame.north east)
-- (frame.south east) -- (frame.south west)
-- ([xshift=-1mm,yshift=-1mm]frame.north west) [sharp corners]-- cycle; },interior engine=empty, }, fonttitle=\bfseries, title= #2,#1}

\newtcolorbox[auto counter,number within=section]{ejemploBoxNumerado}[2][]{enhanced,
colback=white,colframe=green!50,boxrule=0.4mm, attach boxed title to top left={xshift=1cm,yshift*=1mm-\tcboxedtitleheight}, varwidth boxed title*=-3cm, boxed title style={frame code={ \path[fill=tcbcolback!30!black] ([yshift=-1mm,xshift=-1mm]frame.north west) arc[start angle=0,end angle=180,radius=1mm] ([yshift=-1mm,xshift=1mm]frame.north east) arc[start angle=180,end angle=0,radius=1mm]; \path[left color=tcbcolback!60!black,right color=tcbcolback!60!black, middle color=tcbcolback!80!black] ([xshift=-2mm]frame.north west) -- ([xshift=2mm]frame.north east) [rounded corners=1mm]-- ([xshift=1mm,yshift=-1mm]frame.north east)
-- (frame.south east) -- (frame.south west)
-- ([xshift=-1mm,yshift=-1mm]frame.north west) [sharp corners]-- cycle; },interior engine=empty, }, fonttitle=\bfseries, title=Ejemplo~\thetcbcounter. #2,#1}

\newtcolorbox[auto counter,number within=section]{ejercicioBoxNumerado}[2][]{enhanced,
colback=red!5,colframe=red!50,boxrule=0.4mm, attach boxed title to top left={xshift=1cm,yshift*=1mm-\tcboxedtitleheight}, varwidth boxed title*=-3cm, boxed title style={frame code={ \path[fill=tcbcolback!30!black] ([yshift=-1mm,xshift=-1mm]frame.north west) arc[start angle=0,end angle=180,radius=1mm] ([yshift=-1mm,xshift=1mm]frame.north east) arc[start angle=180,end angle=0,radius=1mm]; \path[left color=tcbcolback!60!black,right color=tcbcolback!60!black, middle color=tcbcolback!80!black] ([xshift=-2mm]frame.north west) -- ([xshift=2mm]frame.north east) [rounded corners=1mm]-- ([xshift=1mm,yshift=-1mm]frame.north east)
-- (frame.south east) -- (frame.south west)
-- ([xshift=-1mm,yshift=-1mm]frame.north west) [sharp corners]-- cycle; },interior engine=empty, }, fonttitle=\bfseries, title=Ejercicio~\thetcbcounter. #2,#1}

\newtcolorbox[auto counter,number within=section]{notaNumerada}[2][]{colbacktitle=black!2!white, coltitle=red!70!black,fonttitle=\bfseries,title=Nota~\thetcbcounter. #2,#1,boxrule=0.1mm,colback=black!2}
\setlength{\parindent}{0pt}
\setlength{\parskip}{10pt}
\pagestyle{fancy}


\begin{document}
\begin{titlepage}

\centering

{\scshape\LARGE Navigating the Nexus: Climate Change, Clean Energy, and Nuclear Nonproliferation \par}
\begin{figure}[ht!]
\centering
\vspace{0.5 cm}
\includegraphics[scale=0.25]{images/gt-seal_0.png}
\end{figure}
\vspace{0.39 cm}

\centering

{\itshape\LARGE Summary of the literature review studying the NEXUS of Clean Energy, Nuclear Nonproliferation, and Climate Change.
 \par}
\vspace{1.3cm}
{\scshape\LARGE  Executive Summary \par}
\vspace{0.2 cm}

\vspace{1.5cm}
\begin{figure}[ht!]
\centering
\includegraphics[scale=0.08]{images/Initials.png}
\end{figure}
{\bfseries\LARGE Ethan Masters \par}
\vspace{0.55cm}
{\LARGE \textit{PI: Dr. Kosal \& Dr. Whitlark}\par}
\vspace{1.2cm}

{\Large May - August 2022 \par}
\end{titlepage}

\lhead{Final Undergraduate Research Report}
\rhead{ Ethan Masters }
\renewcommand{\headrulewidth}{0.5pt}

\newpage

\section{Executive Summary}

The intersection of climate change, clean energy, and nuclear nonproliferation is an increasingly pivotal area of global policy discourse. As nations seek sustainable pathways to mitigate climate change, nuclear energy has emerged as both a potential solution and a complex geopolitical challenge. The literature underscores the interdependencies among these domains, revealing both the opportunities and risks associated with expanding nuclear power within the context of clean energy transitions.

The deployment of nuclear energy is frequently framed as an indispensable component of achieving deep decarbonization. Nuclear power provides a stable, low-carbon electricity source that complements renewable energy technologies, particularly in addressing intermittency concerns \cite{bunn2019nuclear}. However, the feasibility of large-scale nuclear expansion remains contested, with economic constraints, regulatory hurdles, and public skepticism serving as persistent barriers \cite{KESSIDES2012185}. While advancements in reactor technology, such as small modular reactors (SMRs), have been proposed as cost-effective and safer alternatives, the extent to which they can overcome existing limitations remains uncertain \cite{DUFFEY2005535}.

Beyond energy considerations, nuclear expansion presents significant security implications. The same technologies that enable nuclear power—uranium enrichment and spent fuel reprocessing—pose inherent proliferation risks. The literature highlights concerns that a renewed nuclear renaissance, particularly in emerging markets, could exacerbate nuclear weapons proliferation or be leveraged by authoritarian regimes for strategic geopolitical influence \cite{Goldschmidt2010multilateral}. Additionally, the transnational nature of the nuclear supply chain raises issues of dependency, as certain state-backed nuclear suppliers, notably Russia and China, are using reactor exports to strengthen diplomatic and economic leverage \cite{Nguyen}.

The governance of nuclear technology and its role in climate policy requires robust international oversight. Existing nonproliferation frameworks, such as the Nuclear Non-Proliferation Treaty (NPT) and International Atomic Energy Agency (IAEA) safeguards, remain essential yet increasingly strained under the pressures of expanding civilian nuclear programs. Proposals for strengthening the nonproliferation regime include multinational fuel-cycle management, enhanced real-time monitoring of nuclear materials, and the development of proliferation-resistant reactor designs \cite{scheffran2015climate}. However, achieving widespread international cooperation on these fronts remains a significant challenge.

Public perception is another crucial dimension in determining the viability of nuclear energy as a climate solution. While some studies indicate that concerns over climate change can enhance public acceptance of nuclear power, this acceptance is often conditional on assurances of safety, waste management, and institutional trust \cite{CORNER20114823}. In certain regions, particularly those with historical opposition to nuclear energy, strong anti-nuclear sentiment persists, further complicating policy pathways \cite{Doyle}.

In sum, the literature suggests that while nuclear energy could play a vital role in decarbonization efforts, its expansion necessitates a holistic approach that integrates energy security, nonproliferation safeguards, and public engagement. The complexities of this nexus highlight the need for interdisciplinary solutions that balance climate imperatives with geopolitical and security realities. Future research should continue to refine policy mechanisms that mitigate proliferation risks while ensuring equitable access to clean energy technologies. 

\newpage

\printbibliography

\end{document}



